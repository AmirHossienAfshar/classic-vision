\documentclass[a4paper,12pt]{article}
\usepackage{fullpage}
\usepackage{amsmath,amsthm,amsfonts,amssymb,amscd}
\usepackage{xcolor}
\usepackage{graphicx}
\usepackage{adjustbox}
\usepackage{geometry}
\usepackage{caption}
\usepackage{xepersian}
\usepackage{multicol}
\usepackage{listings}
\usepackage{color}
\usepackage{hyperref}
\usepackage{bidi} 
\usepackage{enumitem}

% Colors
\definecolor{titlepagecolor}{cmyk}{0.75,0.68,0.67,0.90} % Cover background
\definecolor{CustomAccent}{HTML}{2BAB8C} % Accent color for English text
%\definecolor{CustomBackground}{HTML}{1C1C1C} % Background for content pages
\definecolor{CustomBackground}{cmyk}{0.75,0.68,0.67,0.90}% Background for content pages

%%%%%%%%%%%%%%%%%%%%%%%%%%%%%%%%%%%%%%%%%%%%%%%%%%%%%%%%%%

\definecolor{codebg}{cmyk}{0.75,0.68,0.67,0.90} % same as CustomBackground
\definecolor{accent}{HTML}{2BAB8C} % same as CustomAccent
\definecolor{codegray}{rgb}{0.8,0.8,0.8}
\definecolor{codegreen}{rgb}{0.4,1,0.4}
\definecolor{codepurple}{rgb}{1,0.6,1}
\definecolor{keywordcolor}{rgb}{1,0.3,0.6}

\lstdefinestyle{darkstyle}{
	backgroundcolor=\color{codebg},   
	commentstyle=\color{codegreen},
	keywordstyle=\color{keywordcolor},
	numberstyle=\tiny\color{codegray},
	stringstyle=\color{codepurple},
	basicstyle=\ttfamily\footnotesize\color{white},
	breakatwhitespace=false,         
	breaklines=true,                 
	captionpos=b,                    
	keepspaces=true,                 
	numbers=left,                    
	numbersep=10pt,                  
	showspaces=false,                
	showstringspaces=false,
	showtabs=false,                  
	tabsize=4,
	frame=single,
	rulecolor=\color{accent}
}

\lstset{style=darkstyle}

%%%%%%%%%%%%%%%%%%%%%%%%%%%%%%%%%%%%%%%%%%%%%%%%%%%%%%%%%%%%







% Persian and Latin fonts
\settextfont{Vazir.ttf}[BoldFont = Vazir-Bold.ttf, Path = fonts/]
\setlatintextfont{Times New Roman}

% Line spacing
\renewcommand{\baselinestretch}{1.2}
\renewcommand{\thesection}{\arabic{section})}

\color{white}


% Homework number
\newcommand{\HomeworkNumber}{1}

% Cover-only settings
\pagenumbering{gobble}

% ---------- COVER PAGE ----------
\begin{document}
	\begin{latin}
		\begin{titlepage}
			\newgeometry{top=1in,bottom=1in,right=0in,left=0in}
			\thispagestyle{empty}
			\pagecolor{titlepagecolor}
			\color{white}
			\begin{center}
				\vspace*{\stretch{1}}
				
				{\fontsize{48}{0}\bfseries\selectfont \color{CustomAccent} INTRODUCTION TO COMPUTER VISION}
				
				\vskip 1.5\baselineskip
				{\fontsize{24}{0}\selectfont PROJECT DOCUMENTATION}
				
				\vspace*{\stretch{2}}
				\adjincludegraphics[width=1\paperwidth]{assets/cover2.png}
				
				\vspace*{\stretch{2}}
				{\fontsize{20}{0}\selectfont \color{CustomAccent}
					Ferdowsi University of Mashhad \\
					Department of Computer Engineering
				}
				
				\vskip 1.5\baselineskip
				{\Large SPRING 2025}
				
				\vspace*{\stretch{1}}
			\end{center}
		\end{titlepage}
	\end{latin}
	
	% ---------- RESET PAGE SETTINGS ----------
	\clearpage
	\nopagecolor
	\pagecolor{CustomBackground}
	\color{white}
	\newgeometry{top=1in,bottom=1in,left=1in,right=1in}
	\pagenumbering{arabic}
	
	% ---------- HEADER (PERSIAN) ----------
	\hrule \medskip
	\begin{minipage}{0.295\textwidth}
		\raggedleft \color{CustomAccent}
		مبانی بینایی کامپیوتر\\
		دانشگاه فردوسی مشهد\\
		گروه مهندسی کامپیوتر
	\end{minipage}
	\begin{minipage}{0.4\textwidth}
		\centering 
		\includegraphics[scale=0.3]{assets/fum-logo.png}
	\end{minipage}
	\begin{minipage}{0.295\textwidth} \color{CustomAccent}
		داکیومنت پروژه \\
		دکتر طاهری نیا \\
		بهار 1404
	\end{minipage}
	\medskip\hrule
	\bigskip	
	
	%%%%%%%%%%%%%%%%%%%%%%%%%%%%%%%%%%%%%%%%%%%%%%%%%%%%%%%%%%%%%%%%%%%%%%
	
	\begin{table}[h]
		\centering
		\begin{tabular}{|l|l|}
			\hline
			\textbf{نام و نام خانوادگی} & \textbf{شماره دانشجویی} \\
			\hline
			امیرحسین افشار & 4012262196 \\
			\hline
		\end{tabular}
	\end{table}
	
\hline
%	\begin{figure}[h]
	%		\centering
	%		\includegraphics[scale=0.35]{assets/template.png}
	%		\caption*{\textcolor{CustomAccent}{k-means}}
	%	\end{figure}

	
	\section{سوال اول: معمای داوینچی}
	 در ابتدا برای حل پازل، عملیات denoising را برای تصویر
$processed\_img\_part\_1.jpg$
انجام شد. بدین منظور، از هیستوگرام تصویر بهره گرفته شده است \textsuperscript{[1]} ؛ هیستوگرام تصویر با سطوح خاکستری، بر اساس نرمال کردن میزان پیکسل های با شدت بین 0 تا 255 ساخته می شود تا شکل 
function distribution probability
بدست آید.
 پلات زیر، بیانگر هیستوگرام برای تصویر نویزی شماره 1 است:

\begin{figure}[h]
		\centering
		\includegraphics[scale=0.35]{assets/plot1.png}
		\caption{\textcolor{CustomAccent}{پلات هیستوگرام تصویر نویزی شماره 1}}
	\end{figure}
	
	این شکل، بیان می کند که تعداد پیکسل ها با مقدار نزدیک به صفر و مقدار نزدیک به 255 بسیار زیاد است. همچنین هیستوگرام احتمال قابل انتظار برای تصاویر نویزی با نوع نویز نمک فلفل به شکل زیر است:
	
\begin{figure}[h]
	\centering
	\includegraphics[scale=0.40]{assets/plot2.png}
	\caption{\textcolor{CustomAccent}{هیستوگرام قابل انتظار برای تصاویر نویزی نمک و فلفل}}
\end{figure}

\vfill
\hline
\begin{LTR}
	\begin{latin}
		\begin{center}
			\begin{minipage}{0.9\linewidth}
				\small % or \footnotesize
				\textsuperscript{[1]} Asoke Nath: Image Denoising Algorithms: A Comparative Study of Different Filtration Approaches Used in Image Restoration
			\end{minipage}
		\end{center}
	\end{latin}
\end{LTR}

با مقایسه شکل 1 و 2 می توان نتیجه گرفت که نویز تصویر شماره 1 از نوع نمک و فلفل می باشد. بنابراین، برای denoise کردن تصویر، بهترین گزینه، استفاده از فیلتر median می باشد. 
شایان ذکر است که در عملیات denoise کردن تصویر برای اطمینان بیشتر از نوع نویز، از انواع فیلترها استفاده شد که به شرح زیر هستند:

\begin{table}[ht]
	\centering
	\begin{latin}
	\begin{tabular}{|c|l|l|}
		\hline
		\textbf{Category} & \textbf{Filter Name} & \textbf{Function} \\
		\hline
		Smoothing Filters  & Box Filter              & \texttt{denoise\_box\_filter} \\
		& Gaussian Filter         & \texttt{denoise\_gaussian\_filter} \\
		\hline
		Statistical Filters & Median Filter           & \texttt{denoise\_median\_filter} \\
		& Max Filter              & \texttt{denoise\_max\_filter} \\
		& Min Filter              & \texttt{denoise\_min\_filter} \\
		\hline
		Advanced Filters   & Bilateral Filter        & \texttt{denoise\_bilateral\_filter} \\
		& Non-Local Means         & \texttt{denoise\_nl\_means} \\
		& Wavelet Filter          & \texttt{denoise\_wavelet\_filter} \\
		& Total Variation Filter  & \texttt{denoise\_total\_variation} \\
		\hline
	\end{tabular}
	\end{latin}
	\caption{انواع فیلتر ها و توابع denoising به کار گرفته شده}
\end{table}


همانطور که از شکل 1 انتظار می رفت، بهترین عملکرد خروجی با استفاده از فیلتر میانه یا همان median بدست می آید. در نهایت، برای نهایی کردن بهترین خروجی با استفاده از فیلتر میانه، دو روش پیگیری شد:
\begin{enumerate}
	\item 
	استفاده از چند مرحله فیلتر median با ابعاد یکسان
	\item 
	استفاده از یک فیلتر median با کرنل بزرگتر
\end{enumerate}
روش اول، به طور کلی برای حفظ جزئیات تصویر مناسب تر است؛ زیرا هرگونه کرنل بزرگ  با ریسک از دست دادن جزئیات همراه است. از طرفی، ممکن است که همگرا شدن تصویر پس از اعمال چند بار فیلتر میانه، به کندی پیش رود و حتی برخی نویز ها در تصویر باقی بمانند که این مشکل، در روش دوم وجود ندارد. 

در نهایت، با توجه به مسئله که تنها نیاز است یک متن از تصویر استخراج شود (و حفظ سایر جزئیات دارای اهمیت کمتری است،) از روش دوم استفاده شد.

\begin{figure}[h]
	\centering
	\includegraphics[scale=0.40]{assets/plot4.png}
	\caption{\textcolor{CustomAccent}{شکل نهایی تصویر denoise شده}}
\end{figure}

\pagebreak

همچنین شایان ذکر است که برای درک بهتر نویز، از حوزه ی فرکانس نیز کمک گرفته شد
 \textsuperscript{[1]}  که بتوان نوع نویز را حدس زد و در نهایت با استفاده از فیلتر های notch و یا band-pass و یا band-reject از نویز تصویر کاست. شکل سوم بیانگر این موضوع است.

\begin{figure}[h]
	\centering
	\includegraphics[scale=0.40]{assets/plot3.png}
	\caption{\textcolor{CustomAccent}{تصویر شماره 1 در حوزه فرکانس}}
\end{figure}

با توجه به magnitude تصویر در حوزه فرکانس، نمی توان نویز خاصی را قائل شد که نهایتا، همان فیلتر میانه که از هیستوگرام تصویر درک شده بود، استفاده شد.


\vfill
\hline
\begin{LTR}
	\begin{latin}
		\begin{center}
			\begin{minipage}{0.9\linewidth}
				\small % or \footnotesize
				\textsuperscript{[1]} Digital Image Processing By Gonzalez 2nd Edition 2002, chapter 4.2
			\end{minipage}
		\end{center}
	\end{latin}
\end{LTR}
%
%تبصره: برای کاهش حجم فایل ipynb برای پلاک کردن تصاویر، از figure هایی با ابعاد کوچک استفاده شده است که به ناچار تصاویر را sample down می کند؛ در عین حال، کلیت حل مسئله همچنان صحیح است. همچنین در تابع 
%
%\begin{latin}
%	\begin{lstlisting}[language=Python, caption={extract feature function}]
%def plote_images(images: list[np.ndarray], titles: list[str] = None, cmap: str = 'gray', figsize: tuple = (10, 5))
%...
%	\end{lstlisting}
%\end{latin}
%میتوان با انتخاب figsize بزرگتر، عملکرد هر کدام از فیلتر ها را به شکل واضح تری دید.
\pagebreak
\begin{itemize}
\item 
بخش دوم
\end{itemize}

برای این بخش باید تصویر 
$processed\_img\_part\_2.jpg$
اصلاح می شد. بدین منظور و در ابتدا برای تشخیص نویز، از انرژی و magnetude تصویر استفاده شد؛ زیرا تصویر کاملا پوشیده از نویز بود و پیش بینی نوع نویز را سخت می کرد. همچنین برای درک بهتر کلیت تصویر، با حذف فرکانس های بالا، یک تخمین از کلیت تصویر به دست آمده است. برای این منظور، پلات زیر رسم شده است:
\begin{figure}[h]
	\centering
	\includegraphics[scale=0.60]{assets/plot9.png}
	\caption{\textcolor{CustomAccent}{پیش بینی نوع نویز در تصویر دوم}}
\end{figure}


با دقت کردن به لگاریتم power در پلات (شکل شماره 5) متوجه یک حلقه می شویم که با الگوی سینوسی نویز در تصویر در حوزه مکان تطابق دارد. برای درک بهتر این نویز، به طور مستقل برای تصویر در حوزه فرکانس magnitude و phase رسم شده است که به شکل زیر است:


\begin{figure}[h]
	\centering
	\includegraphics[scale=0.4]{assets/plot10.png}
	\caption{\textcolor{CustomAccent}{تصویر دوم در حوزه فرکانس}}
\end{figure}

که ایده اولیه را تایید می کند. بدین منظور، تنها و به سادگی با ساختن یک فیلتر band-reject در حوزه فرکانس، این مشکل حل شده و تصویر نهایی ساخته می شود.

\pagebreak


\begin{itemize}
	\item 
	بخش سوم
\end{itemize}
برای بخش سوم از پازل، باید تصویر اصلی را از 5 زیر تصویر reconstruct می کردیم. بدین منظور، تصاویر در ابتدا و در کنار یکدیگر plot شده اند:

\begin{figure}[h]
	\centering
	\includegraphics[scale=0.5]{assets/plot5.png}
	\caption{\textcolor{CustomAccent}{زیرتصاویر بخش سوم پازل}}
\end{figure}


همچنین ابعاد هر کدام از تصاویر در جدول زیر بیان شده است:

\begin{table}[h!]
	\centering
	\begin{latin}
	\begin{tabular}{|l|c|c|}
		\hline
		\textbf{Image File} & \textbf{Width (px)} & \textbf{Height (px)} \\
		\hline
		processed\_img\_part\_3\_Level\_0.jpg & 1796 & 1201 \\
		processed\_img\_part\_3\_Level\_1.jpg & 898  & 601  \\
		processed\_img\_part\_3\_Level\_2.jpg & 449  & 301  \\
		processed\_img\_part\_3\_Level\_3.jpg & 225  & 151  \\
		processed\_img\_part\_3\_Level\_4.jpg & 113  & 76   \\
		\hline
	\end{tabular}
	\end{latin}
	\caption{ابعاد زیر تصاویر در بخش سوم پازل}
\end{table}

با توجه به جدول شماره 2 که بیانگر ابعاد پازل است، میتوان متوجه شد که در هر مرحله با تقریب هم طول و هم عرض تصویر نصف می شود. این موضوع، وجود نوعی pyramid را در ذهن تداعی می کند \textsuperscript{[1]}.
 این موضوع، در زیربخش سوم فصل هفتم کتاب گونزالس
 \textsuperscript{[2]}
  به این شکل آمده است:

\begin{figure}[h]
	\centering
	\includegraphics[scale=0.3]{assets/pyramids.png}
	\caption{\textcolor{CustomAccent}{انواع pyramid ها: کتاب گونزالس، زیربخش سوم فصل هفتم}}
\end{figure}




\vfill
\hline
\begin{LTR}
	\begin{latin}
		\begin{center}
			\begin{minipage}{0.9\linewidth}
				\small % or \footnotesize
				\textsuperscript{[1]} medium: A Beginners Guide to Computer Vision (Part 4)- Pyramid \\
				\textsuperscript{[2]} Digital Image Processing By Gonzalez 2nd Edition 2002, chapter 7.3
			\end{minipage}
		\end{center}
	\end{latin}
\end{LTR}
 

\pagebreak

برای درک بهتر هرم، هیستوگرام احتمال برای هر کدام از زیر تصاویر رسم شده است:


\begin{figure}[h]
	\centering
	\includegraphics[scale=0.5]{assets/plot6.png}
	\caption{\textcolor{CustomAccent}{هیستوگرام در هرم}}
\end{figure}

نمودار هیستوگرام نرمال‌شده‌ی هرم، توزیع آماری مقدارهای باقیمانده در هر سطح از هرم را نمایش می‌دهد. در فرآیند ساخت هرم، هر تصویر در یک سطح با نسخه‌ی بزرگ‌ نمایی ‌شده ‌ی تصویر سطح بعدی کم می‌شود، و نتیجه این اختلاف، تصویر باقیمانده‌ای است که تنها حاوی اطلاعات لبه‌ ها، جزییات کوچک است. برای نمایش بهتر این اطلاعات، از یک آفست استفاده شده و همچنین نرمال سازی انجام شده است. در تایید این موضوع، از تبدیل به حوزه فوریه و نمایش فاز هر کدام از زیر تصاویر نیز استفاده شده است:

\begin{figure}[h]
	\centering
	\includegraphics[scale=0.4]{assets/plot7.png}
	\caption{\textcolor{CustomAccent}{هیستوگرام در هرم}}
\end{figure}
 همانطور که مشخص است، لبه های کوچک در سطوح بالا، دارای فاز یکنواخت تری هستند و هرچه به لایه های پایین تر می رویم، این یکنواختی کمتر می شود و تغییرات فرکانس وضوح می یابند.

با دقت به این هیستوگرام، (و همچنین زیرتصاویر ارائه شده در شکل پنجم) میتوان نتیجه گرفت که هرم از نوع لاپلاسی است؛ هر مرحله از downsample شدن تصویر و محاسبه اختلاف با لایه قبلی محاسبه می شود که برای ساختن تصویر اصلی، کافیست معکوس این کار را انجام دهیم.


\pagebreak

برای درست کردن تصویر، در ابتدا تلاش شد که با طی کردن فرایند به شکل معکوس، تصویر اصلی ساخته شود، اما تصویر خروجی شامل کیفیت مناسب نبود. بدین منظور، از تابع resize از کتابخانه cv2 استفاده شد که به صورت خطی تصویر را تغییر سایز می دهد. همچنین، برای بهبود نهایی، مقادیر grayscale به uint8 تغییر یافته اند؛ به عبارتی، مقادیر قبلی سطوح خاکستری با این کار، به بازه (0, 255) گسترش یافتند که باعث بهبود کنتراست شد.
تصویر نهایی ساخته شده نیز در شکل 9 آمده شده است:

\begin{figure}[h]
	\centering
	\includegraphics[scale=0.6]{assets/plot8.png}
	\caption{\textcolor{CustomAccent}{تصویر نهایی ساخته شده از زیرتصاویر}}
\end{figure}


\pagebreak
\section{پروژه دوم: تصاویرنویزی/غیرنویزی}
در این پروژه، هر دو روش یادگیری عمیق و بینایی ماشین کلاسیک پیاده سازی شده است. در ابتدا به تفصیل، روش یادگیری عمیق توضیح داده می شود:

\section{یادگیری عمیق در پروژه دوم}


در ابتدا، برای پیاده سازی سیستم تشخیص نویز ها، این نیاز دیده میشد که دیتاست های در اختیار قرار گرفته ، دیتاست های کوچکی هستند و برای روش های ml و یا dl مناسب نیستند. به گونه ای که حتی با augment کردن داده ها و پیاده سازی بر روی مدل های pre-trained و با fine tune کردن آنها نیز، کفایت نمیکرد.  بنابراین، با بررسی دیتاست های قرار داده شده کشاورزی که از نوع برگ درختان هستند، دیتاست های کاندیدا از وبسایت kaggle به شرح زیر هستند:
\begin{enumerate}
	\begin{LTR}
		\begin{latin}
			\item https://www.kaggle.com/datasets/ichhadhari/leaf-images
			\item https://www.kaggle.com/datasets/mdwaquarazam/agricultural-crops-image-classification/data
			\item https://www.kaggle.com/datasets/alexo98/leaf-detection
		\end{latin}
	\end{LTR}
\end{enumerate}

که طبق بررسی انجام شده، بهترین و مشابه ترین دیتاست، مربوط به دیتاست leaf-detection است. 
دقت شود که این موضوع که دیتاست ها باید برای استفاده این پروژه به شکل کاملی شخصی سازی شوند (به عنوان مثال، نویز به تصاویر اضافه شود) کاملا واضح است و از ابتدا صرفا به دنبال دیتاست هایی صرفا با زمینه مرتبط بودم و سپس دیتاست ها را به شکل مناسبی برای استفاده از این پروژه شخصی سازی کردم. مجددا تاکید میشود که دیتاست انتخاب شده، صرفا انواع برگ های گوناگون بوده است و هیچ هدفی از نویز/دینویز کردن تصاویر در این دیتاست دنبال نشده است. 

بررسی دیتاست آورده شده:

با توجه به این که این دیتاست، صرفا شامل تصاویری از برگ های 10 نوع درخت هندی هستند، بدون توجه به لیبل های آن برگ ها، و صرفا با در نظر گرفتن همه آنها از یک نوع، (همگی را به شکل یکسان و «برگ» در نظر گرفتم) به آنها انواع گوناگونی از نویز ها پیاده سازی کردم. در ابتدا، نویزهای تکی، یعنی :

سپس انواع نویز های دو تایی (تمامی جایگشت ها لحاظ نشده است؛ به علت محدودیت پردازشی گوگل کولب به 16 گیگ حافظه رم، بیش از این نوع نمیتوان جایگشت های مختلفی از نویز در نظر گرفت):

\pagebreak

\begin{table}[h]
	\centering
	\begin{latin}
	\begin{tabular}{|c|c|l|}
		\hline
		\textbf{No.} & \textbf{Combinations} & \textbf{Noise Type} \\
		\hline
		1 & 1 & Gaussian \\
		\hline
		2 & 1 & Salt \& Pepper \\
		\hline
		3 & 1 & Poisson \\
		\hline
		4 & 1 & Speckle \\
		\hline
		5 & 1 & Uniform \\
		\hline
		6 & 2 & Gaussian + Salt \& Pepper \\
		\hline
		7 & 2 & Gaussian + Poisson \\
		\hline
		8 & 2 & Gaussian + Speckle \\
		\hline
		9 & 2 & Gaussian + Uniform \\
		\hline
		10 & 2 & Salt \& Pepper + Speckle \\
		\hline
		11 & 2 & Salt \& Pepper + Uniform \\
		\hline
		12 & 2 & Poisson + Speckle \\
		\hline
		13 & 2 & Poisson + Uniform \\
		\hline
		14 & 2 & Speckle + Uniform \\
		\hline
		15 & 3 & Gaussian + Salt \& Pepper + Speckle \\
		\hline
		16 & 3 & Gaussian + Poisson + Uniform \\
		\hline
		17 & 3 & Salt \& Pepper + Speckle + Uniform \\
		\hline
	\end{tabular}
	\end{latin}
	\caption{لیست انواع نویز های پیاده شده}
\end{table}
	

\pagebreak




	%%%%%%%%%%%%%%%%%%%%%%%%%%%%%%%%%%%%%%%%%%%%%%%%%%%%%%%%%%%%%%%%%%%%%%
	\pagebreak
	
	در ابتدا، به کمک کتابخانه 
	$cv2$
	تابع زیر را برای دریافت فیچرها تکمیل کردیم. 
	
	\begin{latin}
		\begin{lstlisting}[language=Python, caption={extract feature function}]
			def extract_features(image_path)
			# input is an image and the ouput is an array of features.
		\end{lstlisting}
	\end{latin}
	
	


	\begin{figure}[ht]
		\centering
		\begin{minipage}[t]{0.32\textwidth}
			\centering
			\includegraphics[width=\linewidth]{assets/template.png}
			\caption{\textcolor{CustomAccent}{another caption}}
		\end{minipage}
		\hfill
		\begin{minipage}[t]{0.32\textwidth}
			\centering
			\includegraphics[width=\linewidth]{assets/template.png}
			\caption{\textcolor{CustomAccent}{caption new}}
		\end{minipage}
		\vspace{1em}
		\hfill
		\begin{minipage}[t]{0.32\textwidth}
			\centering
			\includegraphics[width=\linewidth]{assets/template.png}
			\caption{\textcolor{CustomAccent}{caption new}}
		\end{minipage}
		\vspace{1em}
	\end{figure}
	
	
	
\clearpage
\section*{منابع}

\begin{LTR}
	\begin{latin}
		\begin{enumerate}[left=0pt,labelsep=5pt,itemsep=0pt,parsep=0pt,topsep=0pt]
			\item Image Denoising Algorithms: A Comparative Study of Different Filtration Approaches Used in Image Restoration.  \url{https://ieeexplore.ieee.org/abstract/document/6524379/}
			\item Digital Image Processing By Gonzalez 4th  \url{https://elibrary.pearson.de/book/99.150005/9781292223070}
			\item Automatic identification of noise in ice images using statistical features \url{https://www.researchgate.net/figure/Simple-pattern-classifier-to-identify-noise-types-of-Gaussian-Speckle-and -Salt-Pepper_tbl1_258714501}
			\item medium: A Beginners Guide to Computer Vision (Part 4)- Pyramid \url{https://medium.com/analytics-vidhya/a-beginners-guide-to-computer-vision-part-4-pyramid-3640edeffb00}
			
			
			% below are dummy
			\item medium: A Beginners Guide to Computer Vision (Part 4)- Pyramid \url{https://medium.com/analytics-vidhya/a-beginners-guide-to-computer-vision-part-4-pyramid-3640edeffb00}
			\item medium: A Beginners Guide to Computer Vision (Part 4)- Pyramid \url{https://medium.com/analytics-vidhya/a-beginners-guide-to-computer-vision-part-4-pyramid-3640edeffb00}
			\item medium: A Beginners Guide to Computer Vision (Part 4)- Pyramid \url{https://medium.com/analytics-vidhya/a-beginners-guide-to-computer-vision-part-4-pyramid-3640edeffb00}
			
		\end{enumerate}
	\end{latin}
\end{LTR}



\end{document}
